\section{Implementation}

The mining pool is written in Django, a web framework for Python which allows for very rapid development. With Django we could focus on the practical side rather than solving unrelated technical difficulties. \\

The mining pool is very lightweight and only spans around 500 lines of code. It consists of two HTTP routes, one for \texttt{/} which miners use, and \texttt{/info} which is a web interface showing pool status. The routes are stored in \texttt{urls.py} and their corresponding endpoints in \texttt{views.py}. Furthermore we wrote two custom classes. The first is \texttt{Block} in \texttt{block.py} for constructing the block, templates and converting various hexadecimal values. The second is \texttt{RPC} in \texttt{rpc.py} for communicating with the wallet. \\

The mining pool uses a SQLite database to store the shares submitted by miners. Its structure is found in \texttt{models.py}, it stores the share timestamp, address of the miner, height of the block, difficulty of the submission and the actual difficulty required by the wallet. \\

The miner starts by sending a POST request to \texttt{/} with the \texttt{getblocktemplate} command. The server then asks the wallet for the current block template, uses that information to construct a \texttt{Block} instance with the same data but a custom coinbase and a much lower difficulty. \\

When the miner has found a solution it sends a POST request to \texttt{/} with the \texttt{submitblock} command. The server then calculates the difficulty of that submission and if it matches or exceeds the network difficulty, the server uses the \texttt{RPC} class to send a \texttt{submitblock} to the wallet, thus successfully mining the block. If the submission difficulty is too low nothing happens, but information about the submission is stored in the \texttt{Shares} SQL table.

\subsection{Using the Mining Pool}

More detailed installation instructions can be found on GitHub. \\

The server can be started with the following command.

\begin{minted}[fontsize=\footnotesize]{text}
$ python manage.py runserver 0.0.0.0:8000
Watching for file changes with StatReloader
Performing system checks...

System check identified no issues (0 silenced).
December 14, 2019 - 02:51:36
Django version 2.2.7, using settings 'smlypool.settings'
Starting development server at http://0.0.0.0:8000/
Quit the server with CONTROL-C.
\end{minted}

The information web page is then available at \texttt{http://localhost:8000/info}. \\

We use an ASIC miner called FutureBit Moonlander 2 to test the mining pool. It comes with a custom build of \texttt{bfgminer}, a popular mining client. We start the miner with the following command, where we pass our payout address as username.

\begin{minted}[fontsize=\footnotesize]{text}
$ bfgminer --scrypt -o http://localhost:8000 -u BEppJqTLw5ByePPbZwm7hByqqwcmsCtVfK -p y -S ALL \
    --set MLD:clock=600 --no-getwork --no-stratum
\end{minted}

The miner should start blinking and soon it outputs something like this.

\begin{minted}[fontsize=\footnotesize]{text}
[2019-12-14 02:50:26] Accepted 000ec771 MLD 0  Diff 67m/10m
[2019-12-14 02:50:34] Accepted 0041d12a MLD 0  Diff 15m/10m
[2019-12-14 02:50:37] Accepted 0038248a MLD 0  Diff 17m/10m
[2019-12-14 02:50:37] Accepted 003ffc0a MLD 0  Diff 15m/10m
\end{minted}

This means that it has found solutions to its shares and is submitting them to the server. We can see these shares in the web interface as well.

\begin{table}[H]
\centering
\small
\begin{tabular}{|l|l|l|l|l|}
\hline
Time                     & Height & Miner                              & Submitted & Actual  \\ \hline
Dec. 14, 2019, 2:50 a.m. & 608802 & BEppJqTLw5ByePPbZwm7hByqqwcmsCtVfK & 0.0156    & 30.6154 \\ \hline
Dec. 14, 2019, 2:50 a.m. & 608802 & BEppJqTLw5ByePPbZwm7hByqqwcmsCtVfK & 0.0178    & 30.6154 \\ \hline
Dec. 14, 2019, 2:50 a.m. & 608802 & BEppJqTLw5ByePPbZwm7hByqqwcmsCtVfK & 0.0152    & 30.6154 \\ \hline
Dec. 14, 2019, 2:50 a.m. & 608802 & BEppJqTLw5ByePPbZwm7hByqqwcmsCtVfK & 0.0677    & 30.6154 \\ \hline
\end{tabular}
\caption{Shares as they appear on the information web page.}
\end{table}

As we can see the submitted difficulty is much lower than what is needed for the network. 

\subsection{Results}

\newpage

\subsection{Code Reference}

\subsubsection{class block.Block()}

This class contains the structure of the current block and includes methods to construct block templates. \\

\texttt{create\_coinbase(outputs)} \\

Builds the coinbase for the block, outputs is a dictionary of addresses and corresponding shares, the shares will be normalized such that the total outputs equal 1000 SMLY. \\

\texttt{create\_gbt(difficulty)} \\

Constructs the \texttt{getblocktemplate} response for miners with the requested difficulty. \\

\texttt{get\_submission\_difficulty(block\_hex)} \\

Calculates the difficulty of the hash of a raw block submission. When miners use submitblock this is used to calculate the difficulty. \\

\texttt{target\_to\_difficulty(target)}\\

Convert the header target field into a difficulty integer. \\

\texttt{difficulty\_to\_target\_hash(difficulty)}\\

Convert difficulty int to a hash target for block template. \\

\subsubsection{class rpc.RPC()}

This class is used to communicate with the SmileyCoin wallet. \\

\texttt{call(cmd, [params])}\\

Sends a command to the wallet with optional parameters and returns the response.
